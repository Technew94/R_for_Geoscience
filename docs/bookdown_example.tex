% Options for packages loaded elsewhere
\PassOptionsToPackage{unicode}{hyperref}
\PassOptionsToPackage{hyphens}{url}
%
\documentclass[
]{book}
\usepackage{amsmath,amssymb}
\usepackage{iftex}
\ifPDFTeX
  \usepackage[T1]{fontenc}
  \usepackage[utf8]{inputenc}
  \usepackage{textcomp} % provide euro and other symbols
\else % if luatex or xetex
  \usepackage{unicode-math} % this also loads fontspec
  \defaultfontfeatures{Scale=MatchLowercase}
  \defaultfontfeatures[\rmfamily]{Ligatures=TeX,Scale=1}
\fi
\usepackage{lmodern}
\ifPDFTeX\else
  % xetex/luatex font selection
\fi
% Use upquote if available, for straight quotes in verbatim environments
\IfFileExists{upquote.sty}{\usepackage{upquote}}{}
\IfFileExists{microtype.sty}{% use microtype if available
  \usepackage[]{microtype}
  \UseMicrotypeSet[protrusion]{basicmath} % disable protrusion for tt fonts
}{}
\makeatletter
\@ifundefined{KOMAClassName}{% if non-KOMA class
  \IfFileExists{parskip.sty}{%
    \usepackage{parskip}
  }{% else
    \setlength{\parindent}{0pt}
    \setlength{\parskip}{6pt plus 2pt minus 1pt}}
}{% if KOMA class
  \KOMAoptions{parskip=half}}
\makeatother
\usepackage{xcolor}
\usepackage{longtable,booktabs,array}
\usepackage{calc} % for calculating minipage widths
% Correct order of tables after \paragraph or \subparagraph
\usepackage{etoolbox}
\makeatletter
\patchcmd\longtable{\par}{\if@noskipsec\mbox{}\fi\par}{}{}
\makeatother
% Allow footnotes in longtable head/foot
\IfFileExists{footnotehyper.sty}{\usepackage{footnotehyper}}{\usepackage{footnote}}
\makesavenoteenv{longtable}
\usepackage{graphicx}
\makeatletter
\def\maxwidth{\ifdim\Gin@nat@width>\linewidth\linewidth\else\Gin@nat@width\fi}
\def\maxheight{\ifdim\Gin@nat@height>\textheight\textheight\else\Gin@nat@height\fi}
\makeatother
% Scale images if necessary, so that they will not overflow the page
% margins by default, and it is still possible to overwrite the defaults
% using explicit options in \includegraphics[width, height, ...]{}
\setkeys{Gin}{width=\maxwidth,height=\maxheight,keepaspectratio}
% Set default figure placement to htbp
\makeatletter
\def\fps@figure{htbp}
\makeatother
\setlength{\emergencystretch}{3em} % prevent overfull lines
\providecommand{\tightlist}{%
  \setlength{\itemsep}{0pt}\setlength{\parskip}{0pt}}
\setcounter{secnumdepth}{5}
\usepackage{booktabs}
\ifLuaTeX
  \usepackage{selnolig}  % disable illegal ligatures
\fi
\usepackage[]{natbib}
\bibliographystyle{apalike}
\IfFileExists{bookmark.sty}{\usepackage{bookmark}}{\usepackage{hyperref}}
\IfFileExists{xurl.sty}{\usepackage{xurl}}{} % add URL line breaks if available
\urlstyle{same}
\hypersetup{
  pdftitle={R for Geoscience},
  pdfauthor={Shuxin Ji},
  hidelinks,
  pdfcreator={LaTeX via pandoc}}

\title{R for Geoscience}
\author{Shuxin Ji}
\date{2023-08-13}

\begin{document}
\maketitle

{
\setcounter{tocdepth}{1}
\tableofcontents
}
\hypertarget{about}{%
\chapter*{About}\label{about}}
\addcontentsline{toc}{chapter}{About}

Hello, this is shuxin's first attempt at harnessing R for geographical data processing.

Before delving into R, my approach primarily involved utilizing desktop software like ArcGIS, QGIS, ENVI, and SNAP to perform tasks such as locating, downloading, and preprocessing geographical data. The advantage of this method lies in its ``what you see is what you get'' nature, where I could anticipate the approximate outcomes of each step, offering a clear mental picture. However, the drawback is evident: any oversight in a step could potentially impact the final result, leading to questionable outcomes. This necessitates starting over, reiterating the data processing, which significantly dampened my work enthusiasm. I believe you've likely experienced a similar process.

The core focus of this book primarily revolves around potential challenges in geographical work. It could also be viewed as a summary of issues I encountered during my doctoral research phase. The primary tool utilized throughout is R. Rather than showcasing intricate code, my aim here is for you to follow the book's methodology, practicing step by step. When you can effortlessly tackle the array of issues described in the book, I'm confident you'll already have transformed into a competent geographical professional.

I believe that the following qualities are required to be a good geographic data processor.

\begin{itemize}
\tightlist
\item
  Mastery of the fundamentals of geography, including earth science, geographic information science, cartography, geostatistics, etc.
\item
  Mastery of data analysis skills, such as statistical analysis, data visualisation, machine learning, etc.
\item
  Familiarity and proficiency with geographic data processing tools R, Python, etc.
\item
  Experience with Linux systems and ability to monitor systems using a shell.
\end{itemize}

\hypertarget{ux8bfeux7a0bux5b89ux6392}{%
\section*{\texorpdfstring{\textbf{课程安排}}{课程安排}}\label{ux8bfeux7a0bux5b89ux6392}}
\addcontentsline{toc}{section}{\textbf{课程安排}}

\hypertarget{practice}{%
\section*{\texorpdfstring{\textbf{practice}}{practice}}\label{practice}}
\addcontentsline{toc}{section}{\textbf{practice}}

  \bibliography{book.bib,packages.bib}

\end{document}
